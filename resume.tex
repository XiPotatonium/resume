% !TEX program = xelatex
% This is my resume
% by nicekingwei

\documentclass{resume}

\usepackage{lastpage}
\usepackage{fancyhdr}
\usepackage{linespacing_fix} % disable extra space before next section

\begin{document}
\pagestyle{fancy}
\fancyhf{}
\renewcommand\headrulewidth{0pt}


\name{Shuhui Wu}

\basicInfo{
  \email{xipotatonium@outlook.com} \textperiodcentered\
  \phone{(+86) 152-6781-1210} \textperiodcentered\
  \github[XiPotatonium]{https://github.com/XiPotatonium}
}

\section{\faGraduationCap\ Education}
\datedsubsection{\textbf{Zhejiang University},  Bachelor}{Sep, 2017 -- Present}
  Computer Science and Technology
  
  GPA 3.90/4.00

\section{\faUsers\ Experience}

\datedsubsection{\textbf{Student Research Training Project of ZJU}}{June, 2019 -- Present}
\role{participator}{}
\begin{itemize}
  \item Took part in a project called Maestro, which is an implementation of the AlphaZero algorithm in Gomoku using LibTorch.
  \item Implemented Monte-Carlo Tree Search for Gomoku and did part of research work for Monte-Carlo Graph Search.
  \item Performed some experiments on the performance of the search algorithm.
\end{itemize}

\datedsubsection{\textbf{Semilat Technology}}{Sep, 2019 -- Present}
\role{co-founder, client developer}{}
\begin{itemize}
  \item Took part in the client development of Yunshan BoQ, which is a automation tool designed for building-cost appraiser.
  \item Proficient in using C\# and WPF, familiar with client development.
  \item Accumulated experience in team communication and large-scale project development.
\end{itemize}

\section{\faGithubAlt\ Projects}

\datedsubsection{\textbf{XiLang} , C\#}{}
XiLang is a toy C\# like OO programming language.
\begin{itemize}
  \item Implement tokenizer and parser using regular expression and top down parsing
  \item Serialize abstract syntax tree to json and visualize using D3.js
  \item Implement a stack based virtual machine called XiVM which can execute byte code of XiLang
\end{itemize}

\datedsubsection{\textbf{MiniSQL} , C++}{}
MiniSQL is a simple database.
\begin{itemize}
  \item Get SQL as input, construct tree-like expression and execute
  \item Implement scanner, which can process data like a stream or pipeline
\end{itemize}

\datedsubsection{\textbf{XISH} , C, Linux}{}
XISH is a simple command-line shell.
\begin{itemize}
  \item Implement various built-in commands like echo, cd, pwd, time.
  \item Pipe, redirection, background, suspension and script are supported.
\end{itemize}

\section{\faCogs\ Skills}
\begin{itemize}[parsep=0.25ex]
    \item \textbf{Programming Language}:
        Proficient: C\#, C, Python. Familiar: C++, Java

    \item \textbf{.Net and Client Development}:
        Proficient in WPF framework and have some experience in UWP development

    \item \textbf{Machine Learning}:
        Understand the basic theory of neural network. Understand the basic idea of taxonomy extraction in NLP.
        Implemented garbage classification, TextCNN, TextGCN and AdversarialRNN using PyTorch,
        visualization of training log using PyPlot.
    
    \item \textbf{Embedded and Hardware}:
        Understand the implementation of MIPS pipeline CPU using FPGA.
        Interest in raspberry pi development and have some experience in STM32F103 development.
\end{itemize}

\section{\faInfo\ More}
\begin{itemize}[parsep=0.25ex]
  \item My blog: \url{https://xipotatonium.github.io/}
  \item Language: English - Fluent (CET-6 604)
  \item PAT Advanced(2020 Spring): Score 96/100, rank 100/2280.
  \item Have a wide range of interests like botany, astronomy, history, writing, singing. 
        Curious about everything, passionate about seeking novelty.
\end{itemize}

\end{document}

